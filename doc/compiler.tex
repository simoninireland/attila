\chapter{The compiler}
\label{chap:compiler}

\begin{precis}
  In this chapter we discuss the Attila Forth compiler word set. We
  cover the user-level words generally encountered in ``normal'' code,
  and then look at the words used to implement the core compiler
  functionality.
\end{precis}


\section{The colon-compiler}
\label{sec:colon-compiler}


\begin{defword}[colon]{:}{``name'' -- xt}
  Define a new word \arg{name} and switch into compilation mode.
\end{defword}

\begin{defword}[semi]{;}{xt --}
  Complete the definition of a word begun by \word{:} and enter
  interpretation mode.
\end{defword}

\begin{defword}[open square]{[}{--}
  Enter interpretation mode.
\end{defword}

\begin{defword}[close square]{]}{--}
  Enter compilation mode.
\end{defword}

\subsection{No-name definitions}
\label{sec:colon-noname}

It's sometimes useful to define ``anonymous'' words that can't be
referred to by name. At the very least these avoid cluttering-up the
namespace, but they can be stored in variables, hung on hooks or used
for building other structures where they won't be called or compiled
in the usual way.

\begin{defword}{colon no name}{:NONAME}{-- xt xt}
  Begin the definition of a new anonymous word and enter compilation
  mode.
\end{defword}

A typical idiom for no-name definitions might be:

\begin{verbatim}
:NONAME ( -- FALSE )
    SOME-INTERESTING-THING EXECUTE
    FALSE ;
HANG-ON INTERESTING-HOOK
\end{verbatim}

\noindent The word is defined anonymously. It's xt is left on the
stack after the closing \word{;} and so can be used by the following
code (\texttt{HANG-ON INTERESTING-HOOK}) that hangs the word onto a
hook that will presumably be called at some point to run a string of
words built at run-time.

%%% Local Variables: 
%%% mode: latex
%%% TeX-master: "attila"
%%% End: 
