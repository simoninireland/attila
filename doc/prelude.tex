
\chapter{The standard prelude}
\label{chap:standard-prelude}

Attila is designed to be minimal at start-up, and to load its main
features as required from source code. This reduces the amount of
bound-in code and encourages experimentation -- at the expense of
making the system load slightly slower than it might do otherwise.

The standard system loads a standard prelude, \texttt{prelude.fs},
which simply includes the source files making up a ``standard'' Forth
language system. Many of these features are essential for a system
that's going to define more code, since it includes things like
control structures and conditionals -- which are strictly unnecessary
to \emph{run} code, although necessary to \emph{build} it.

The standard prelude include a number of elements, including:

\begin{itemize}
\item Conditionals and loops
\item Data words
\item Variables, constants and values
\item Character and string-handling, include ``long'' strings
\item Formatted numeric output
\item Records
\item File I/O and redirection
\item Flexible include paths
\item Word lists
\item Dynamic memory management
\end{itemize}

%%% Local Variables: 
%%% mode: latex
%%% TeX-master: "attila"
%%% End: 
