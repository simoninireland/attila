
\chapter{Introduction}
\label{chap:introduction}

\section{Notational conventions}
\label{sec:conventions}

\subsection{Stack effect comments}
\label{sec:stack-comments}

Forth is a stack-based language, and so the effect that a word has on
the language's stacks is vital to understanding how the word
works. Forth uses a standard format for describing stack effects: a
comment that describes the stack before and after the word's
execution. The stack comments for all the words in the system is given
along with their definitions, presented in a standard described below
(section~\ref{sec:word-definitions}).

The most common stack comment describes the use of the data stack, and
looks like:

\begin{center}
  \texttt{( \textit{a b -- c} )}
\end{center}

\noindent This stack comment describes the stack affect of a word that
expects two values on the data stack, \arg{a} and \arg{b}, with
\arg{b} being the top item, and performs some action that leaves a
single value \arg{d} on the stack afterwards, Crucially, by convention
\emph{no elements farther down the stack are affected}, so the stack
comment describes all there is to know about what the word reads off
the stack and pushes back onto it.

If a word affects the return stack, it will have an additional stack
comment such as:

\begin{center}
  \texttt{( \textit{a --} ) R:( -- a)}
\end{center}

\noindent where the stack comment prefixed with \texttt{R:} describes
the effect on the return stack in the same way as a normal stack
effect describes the data stack. Systems with other stacks may have
other comments as well, for example \texttt{F:} for the floating-point
stack.

A small number of words read the next word or words in the input
stream, for example to create a named word. This is captured in the
stack comment as though the element read was on the stack: the comment
for the word \word{CREATE} for example is:

\begin{center}
  \texttt{( \textit{``name'' --} )}
\end{center}

\noindent indicating that, when executed, it reads a word from the
input stream.

A small number of words also have ``alternative'' stack effects, in
which they may change the stack in two or more different ways. A good
example of this is the word \word{FIND}, which looks for the
definition of a word and has two different effects depending on
whether it finds it. Its stack effect is:

\begin{center}
  \texttt{( \textit{addr n -- xt -1 | 0} )}
\end{center}

\noindent Here the two possible after-effects of the word are
\arg{xt -1} \emph{or} \arg{0}. A user of this word will check the top
value and then know whether there's a second value there or not.

\subsection{Word definitions}
\label{sec:word-definitions}

Stack comments are only partial documentation of a word, of course,
and to understand what they do you need to refer to their full
definition. In this book all words are described using a standard
format. For example the definition of the word \word{DUP} is:

\begin{defword}[dupe]{DUP}{a -- a a}
  Duplicate the cell on the top of the stack.
\end{defword}

The top line consists of the Forth name of the word (\word{DUP}) and
its stack effect as described above
(section~\ref{sec:stack-comments}), which may include effects for
multiple stacks (although in the case of \word{DUP} it's only the
ordinary data stack that's affected). If the pronunciation of the word
isn't obvious, a clue will be given -- well, a clue for a native UK
english-speaker, anyway! -- in quotes at the right. The body of the
definition is a description of the word's effect \emph{from the
  perspective of the user}.

\subsection{Word names}
\label{sec:word-names}

A Forth system consists of a \emph{lot} of words, and some naming
conventions are handy to help spot what's going on. The following
conventions are used in Attila:

\begin{center}
  \begin{tabular}[c]{|p{.2\columnwidth}|p{.5\columnwidth}|p{.2\columnwidth}|}
    \hline
    Name style & Meaning & Examples \\
    \hline\hline
    Ending in \texttt{!} & Store into memory & \word{!} \word{XT!} \\
    \hline
    Ending in \texttt{@} & Fetch from memory & \word{@} \word{C@} \\
    \hline
    Ending in \texttt{,} & Compile to memory & \word{,} \word{COMPILE,}
    \\
    \hline
    Starting with a ``type name'' & A word that operates on that type &
    \word{C@} \word{XT!} \\
    \hline
    Round brackets & An ``inner'' word, used for defining something but
    not intended for general use & \word{(FIND)} \word{(START)} \\
    \hline
    Square brackets & ``Immediate'' words that execute even when in
    compilation mode & \word{[IF]} \word{[COMPILE]} \\
    \hline
  \end{tabular}
\end{center}

It's worth repeating that these are \emph{conventions}, not
\emph{rules}: they aren't used completely consistently, and may
mislead (although we try not to). Certainly not all immediate words
appear in square brackets, for example.


\section{A brief history of Forth}
\label{sec:history}

\ldots

One of the first book-length references to Forth comes from
\citet{TIL}, who describes the implementation of a Forth compiler from
the ground up in Intel 8080 machine code. The system is so compact
that it will fit comfortably into 4k of memory -- quite an
accomplishment for an interactive system! Some of the memory savings
seem quite inspired in hindsight, but would have been regarded as
pass\'e for modern systems -- until very recently when the increased
use of wireless sensor networks raised to profile of programming on
highly resource-constrained platforms.  


%%% Local Variables: 
%%% mode: latex
%%% TeX-master: "attila"
%%% End: 
