\chapter{The cross-compiler}
\label{chap:cross-compiler}

\begin{precis}
  This chapter describes the Attila cross-compiler, which is the
  mechanism by which Attila is compiled, extended and ported to other
  systems. We start with what cross-compilation is and why it's a good
  idea, and then show how you can use the cross-compiler to build new
  Attila virtual machines or standalone programs. This is the basis
  for embedding Attila that we'll look at in
  part~\ref{part:embedded-systems}.
\end{precis}

\section{What is cross-compilation?}
\label{sec:what-is-cross-compilation}

A compiler is a tool that takes a program written in one language (the
\emph{source} language) and translates it into an equivalent program
in another language (the \emph{target} language). Often the target
language is the machine code of a particular processor, in which case
we have a \emph{native-code compiler}: another popular choice is to
target C, which is then run through a C compiler to generate
executable machine code.

This raises the question of what language to write the compiler itself
in. There are a number of popular choices: C, obviously, but also
Haskell and ML to benefit from the improved expressive power one gets
from functional languages. Of course using languages like this means
that the compiler-writer needs to work in at least two languages: the
language being compiled, and the language in which the compiler is
written (and possibly another target language too). Sometimes this is
useful, sometimes not: it certainly adds to the cognitive load of
compiler-writing.

because a compiler generates code, it targets a particular kind of
computer (the target computer). Often the compiler runs on the same
kind of machine that it targets: often exactly the same machine. For
modern server, desktop and laptop computers that's fine, but it works
less well for compiling code for an embedded device like a washing
machine: compilers are often large pieces of code, too bit for a
washing machine to run itself. In this case it's common for the
compiler to run on a different machine (the \emph{host} machine) and
generate code for the target. This is important, because the host may
not even be able to execute the code that the compiler generates for
the target. This technique -- of using one machine to build code that
runs on another -- is called \emph{cross-compilation} and is
accomplished using a \emph{cross-compiler}.


\section{What is bootstrapping?}
\label{sec:what-is-bootstrapping}

One of the disadvantages of the compilers we've just described is that
they require programmers to know several languages. In many cases it'd
be more attractive to focus on \emph{one} language, and if that
language is sufficiently powerful and expressive to write a compiler in,
we could use it to write a compiler \emph{for itself}: the host
language would be the same as the source language, and potentially
might also be the same as the target language. That way we only need
to learn a single language to write our compiler.

A moment's thought will show the problem with this: what compiler do
we start off using, if we haven't got a compiler for it. Let's use C
as an example. We can write a C compiler in C, and then compile this
compiler using \emph{another} C compiler. If we have a new machine
that needs a C compiler, we can use a C cross-compiler to generate a
compiler for that target machine. You can use this trick several
times, cross-compiling compilers to new platforms.

Of course someone had to write an initial compiler from scratch, which
would be hard and probably involve assembler, so it makes sense to
avoid doing this as far as possible. So suppose we create a simple
language that's \emph{just about} expressive enough to write a
compiler in, and hand-craft a compiler for that language. We then
write the compiler for our real target language in this minimal
language, and use our hand-crafted compiler to build a compiler for
the full language. This approach is called \emph{bootstrapping}: we
build an initial, bootstrap compiler and use this to compile the
``real'' compiler. The bootstrap compiler will typically be simple and
inefficient, but that's fine: the \emph{next} compiler, the one we use
the bootstrap compiler to build, can be full-featured and flexible. By
doing this we minimise the work of writing a compiler in a strange
host language, and maximise the use of our target language. In effect
the target language compiler is written in itself -- with a bit of
magic to start the process off.


\section{The Attila cross-compiler}
\label{sec:attila-cross-compiler}

Attila is based around a cross-compiler for Forth. The source and host
languages are both Forth: \emph{everything} is written in Forth,
including the compiler and cross-compiler.

In the initial installation of the system, of course, we can't use
Attila to compile itself since we don't have an Attila system to work
from. We therefore have a small bootstrap compiler,
\texttt{bootattila}, that's written in C and implements a limited
sub-set of Forth, adequate for writing a cross-compiler in. We then
use this bootstrap compiler to load a Forth cross-compiler, which can
then load and compile the ``real'' Attila system, \texttt{attila},
from its source code, which is written in Forth. Once that's done we
can forget about \texttt{bootattila}, because any time we want to
re-build, port or amend the Attila compiler we can do so from within
Attila, in Forth.

Actually it's not \emph{quite} that simple. Some of the really
low-level parts of Attila are too low-level to be written in
Forth. The base level of Attila is therefore written in some other
base target language -- typically C, but also possible assembler for a
particular platform. The Attila cross-compiler generates code for this
target language, which is then compiled to build a new \texttt{attila}
binary. The reason this is a \emph{cross}-compiler is that the C
compiler (or assembler) that we use to build the final binary may not
target the same system that we're running the cross-compiler on. The
cross-compiler can therefore build Attila run-times that are intended
for other systems, specifically for embedded systems -- a topic we
return to in part~\ref{part:embedded-systems}.

\subsection{Cross-compiler basics}
\label{sec:cross-compiler-basics}

A Forth compiler is simply a word set that is designed to create new
words (see section\ref{sec:compiler}). The colon-compiler is the
archetypal version: it takes Forth program source code of the form:

\begin{verbatim}
: 2* ( n -- m )
    2 * ;
\end{verbatim}

\noindent and generates a runnable definition of the given word
(\word{2*} in this case).  To do this it uses a combination of the
defining word \word{:}, the outer executive, and some low-level words
like \word{(WORD)} and \word{COMPILE,} to create the executable
definition and its code in a form that Attila can execute.

A Forth \emph{cross}-compiler is conceptually very simple. Like the
compiler, it reads source code and ``compiles'' it. Unlike the
compiler, it generates code in a form that \emph{can't} be directly
executed by can be used to \emph{generate} executable code at a later
stage. To do this it simply re-defines the compiler word set so that
the compilation process can proceed as exactly as normal, except for
the fact that the words produce different code. The cross-compiler
consists of a modified version of the normal Forth compiler that
intercepts all the words used in compilation and re-defines them to
perform cross-compilation instead, plus some housekeeping functions to
do with recording and saving the cross-compilable code that's generated. 


\section{Using the cross-compiler}
\label{sec:using-cross-compiler}

The cross-compiler has two main uses:

\begin{enumerate}
\item\label{enum:using-cross-compiler-dev} It can be used to
  \emph{re-create an Attila development system}, perhaps for a new
  machine or to add new features into the virtual machine image.
\item\label{enum:using-cross-compiler-standalone} Alternatively, it
  can be used to \emph{create standalone Attila programs} that aren't
  intended to be interactive Forth systems but instead perform some
  dedicated function. This is more like ``normal'' programming, in the
  sense that the final product isn't intended to be a development
  system.
\end{enumerate}

Option~\ref{enum:using-cross-compiler-dev} is used when Attila is
first installed, and can be used to customise the ``standard'' system
with new features. Option \ref{enum:using-cross-compiler-standalone}
is used to build embedded systems, as well as stand-alone programs
that to all intents and purposes are independent of having been
written using Attila, and can be distributed and executed without
further access to an Attila compiler. 

\subsection{Re-building the standard system}
\label{sec:cross-compiler-rebuilding}



\subsection{Driver files}
\label{sec:cross-compiler-driver-file}

\subsubsection{The standard driver}
\label{sec:cross-compiler-standard-driver}

\subsubsection{Custom drivers}
\label{sec:cross-compiler-custom-drivers}



 


%%% Local Variables: 
%%% mode: latex
%%% TeX-master: "attila"
%%% End: 
